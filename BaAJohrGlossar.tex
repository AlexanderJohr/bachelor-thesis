\section*{Glossar}
\label{sec:Glossar}

\begin{tabularx}{\textwidth}{p{0.25\textwidth} X}
QlikView & eine Business Intelligence-Software des Softwareunternehmens Qlik zur sekundenschnellen Analyse von Geschäftsdaten \\
Qlik Sense & eine Business Intelligence-Software des Softwareunternehmens Qlik, die als Nachfolger für QlikView verstanden werden kann \\
Sheet Object & positionierbare Objekte in QlikView und Qlik Sense, die zur Visualisierung von Daten mithilfe von Diagrammen, Tabellen und Listen eingesetzt werden\\
Document Extension & eine Erweiterung, die das Verhalten bzw. das Aussehen eines QlikView Dokumentes manipulieren kann\\
Extension Object & eine Erweiterung für QlikView und Qlik Sense die sich ähnlich verhält wie ein Sheet Object. Eine solche Erweiterung erhält Daten der Anwendung, welche von ihr visualisiert und zur Selektion angeboten werden können\\
HTML & Die Hypertext Markup Language ist eine textbasierte Auszeich\-nungs\-sprache zur Strukturierung von Webseiten\\
CSS & Die Cascading Style Sheets dienen der textbasierten Erstellung von Darstellungsregeln von beispielsweise HTML-Dokumenten und helfen Darstellung und Strucktur der Inhalte zu trennen\\
JavaScript & Eine Skriptsprache für das Erstellen von Webanwendungen \\
JSON & Die JavaScript Object Notation ist ein textbasiertes Datenformat zum Datenaustausch zwischen Anwendungen \\
XML & Die Extensible Markup Language ist eine textbasierte Auszeich\-nungs\-sprache für hierarchisch strukturierte Daten\\
Dart & Eine Programmiersprache für das Erstellen von Anwendungen, die im Browser und auf dem Server mit integrierter Dart-VM lauffähig sind.  \\
Dart2js & Ein Transcompiler zur Übersetzung von Dart-Quellcode nach JavaScript um in Dart entwickelte Webanwendungen auch im Browser ohne Dart-VM lauffähig zu machen  \\
\end{tabularx}