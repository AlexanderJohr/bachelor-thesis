
\newpage
\thispagestyle{empty}

Hochschule Harz\newline
Fachbereich Automatisierung und Informatik

\begin{center}

\large{\textsc{Thema und Aufgabenstellung der Bachelorarbeit}}


\large{\textsc{BA AI 60/2014}}

\vfill

\large{\textsc{für Herrn Alexander Johr}}

\vfill

\vfill

\Large{\textsc{Entwicklung von Extension Objects sowie Document Extensions für QlikView 11 und Qlik Sense mit Dart}}

\end{center}

\vfill

Die Produkte QlikView und Qlik Sense - Business Intelligence Software des Software\-unter\-nehmens Qlik - bieten ihren Anwendern mit einer Reihe an unterschiedlichen Diagramm\-typen einen Überblick über ihre Geschäftsdaten mittels Ad-hoc-Analysen. Nicht alle Wünsche der Anwender lassen sich über die Konfigurations\-möglich\-keiten dieser vorgefertigten Diagramme abdecken. Eine Alternative stellen die sogenannten Extension Objects und Document Extensions dar, die mit mithilfe von Webtechnologien wie JavaScript, HTML, CSS und XML entwickelt werden können. Die Entwicklung von Programmen mit JavaScript erweist sich jedoch gegenüber der Entwicklung mit anderen Programmiersprachen als sehr mühsam.


Ziel dieser Bachelorarbeit ist es die Entwicklung dieser Extension Objects sowie Document Extensions mit der Programmiersprache Dart von Google umzusetzen, da sich diese für die Entwicklung skalierbarer und strukturierter Webanwendungen eignet. Die Arbeit bietet einen Leitfaden wie solche Extensions mit Dart entwickelt werden können und welche Vor- und Nachteile dies gegenüber der Entwicklung mit JavaScript bietet. 


Die Bachelorarbeit beinhaltet folgende Teilaufgaben:
\begin{itemize}
	\itemsep0em
	\item Analyse der Unterschiede von QlikView 11 und Qlik Sense bei der Entwicklung von Extension Objects sowie von Document Extensions
	\item Analyse der Einschränkungen von Extensions Objects gegenüber der von QlikView 11 und Qlik Sense mitgelieferten Sheet Objects
	\item Analyse der Auswirkungen der Extensions auf die Performance
	\item Analyse der Vor- und Nachteile der Entwicklung von Extensions mit Dart im Vergleich zur Entwicklung von Extensions mit JavaScript
	\item Entwicklung von zeitsparenden Methoden zur Entwicklung von Extensions
	\item Bewertung der Ergebnisse
\end{itemize}

\begin{tabularx}{\textwidth}{@{} *2{>{\centering\arraybackslash}X}@{}}
Prof. Jürgen Singer Ph.D. & Prof. Daniel Ackermann \\
1. Prüfer                 & 2. Prüfer	 \\
\end{tabularx}	     